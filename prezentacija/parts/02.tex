% TODO skratiti na 5 slajdova


\begin{frame}{Izazovi}
    \begin{itemize}
        \item Sa visokog nivoa, problem sinteze se može razložiti na dva potproblema:
            \begin{itemize}
                \item Definisanje specifikacija željenog programa
                \item Pretraživanje prostora mogućih programa u potrazi za onim koji zadovoljava definisane specifikacije
            \end{itemize}
        \item Prostor programa se povećava eksponencijalno brzo u odnosu na veličinu željenog programa
    \end{itemize}
\end{frame}


\begin{frame}{Izazovi - Definisanje specifikacija}
    \begin{itemize}
        \item Većina programa koji se danas koriste su previše komplikovani da bi se u potpunosti opisali bilo formalnim bilo neformalnim metodama
        \item Potrebno je omogućiti korisniku da definiše željeni program do neke tačke, a da kasnije tokom sinteze, interaktivno sa računarom, postepeno dolazi do rešenja
        \item \emph{FlashFill}
    \end{itemize}
\end{frame}

\begin{frame}{Izazovi - Pretraživanje prostora programa}
    \begin{itemize}
        \item Prostor programa - skup koji sadrži sve moguće programe koji se mogu napisati
        \item Pretraga ovog skupa znači nalaženje programa koji zadovoljava specifikacije
        \item Tehnike pretrage se mogu zasnivati na:
            \begin{itemize}
                \item Enumerativnoj pretrazi
                \item Dedukciji
                \item Tehnikama sa ograničenjima
                \item Induktivnim i statističkim metodama
            \end{itemize}
    \end{itemize}
\end{frame}

\begin{frame}{Izazovi - Pretraživanje prostora programa - Enumerativna pretraga}
    \begin{itemize}
        \item Jedna od najefikasnijih tehnika za generisanje malih programa
        \item Tehnike \emph{čišćenja}
        \item Prvo se na neki način opiše prostor pretrage u kome se nalazi željeni program
        \item Kada se mogući programi numerišu po osobinama, mogu da se odmah odbace oni koji ne zadovoljavaju specifikaciju
        \item Enumerativna tehnika je poluodlučiva
    \end{itemize}
\end{frame}

\begin{frame}{Izazovi - Pretraživanje prostora programa - Deduktivna pretraga}
    \begin{itemize}
        \item Pretpostavka da postoji celokupna formalna specifikacija željenog programa
        \item Rešenje se sintetiše postupkom dokazivanja teorema, logičkim zaključivanjem i razrešavanjem ograničenja
        \item Deduktivna pretraga je pretraga odozgo nadole
        \item Koristi tehniku podeli-pa-vladaj
        \item Deljenje problema na potprobleme nije moguće u opštem slučaju
        \item Kombinovanje deduktivne pretrage sa enumerativnom
    \end{itemize}
\end{frame}

\begin{frame}{Izazovi - Pretraživanje prostora programa - Tehnike sa ograničenjima}
    \begin{itemize}
        \item Tehnike prilagođavanja datim ograničenjima
        \item Dva velika koraka:
            \begin{itemize}
                \item Generisanje ograničenja
                \item Razrešavanje ograničenja
            \end{itemize}
    \end{itemize}
\end{frame}


\begin{frame}{Izazovi - Pretraživanje prostora programa - Statistička pretraga}
    \begin{itemize}
        \item Koriste neku vrstu statistike kako bi došle do rešenja
        \item \emph{Mašinsko učenje}
        \item \emph{Genetsko programiranje}
        \item \emph{Probabilističko zaključivanje}
    \end{itemize}
\end{frame}


\begin{frame}{Izazovi - Pretraživanje prostora programa - Induktivna pretraga}
    \begin{itemize}
        \item Može se smatrati kao nadogradnja tehnike pretrage sa ograničenjima
        \item Prilikom svake iteracije se generišu ograničenja
        \item Rešavačem se dođe do mogućeg rešenja a zatim se ispita da li je ono zadovoljavajuće kao opšte rešenje
        \item Može da koristi tehnike mašinskog učenja - \emph{Aktivno učenje}
        \item \emph{CEGIS}
    \end{itemize}
\end{frame}
