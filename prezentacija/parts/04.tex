
\begin{frame}{CEGIS - Sinteza vodjena uzorom}
    \begin{itemize}
        \item \emph{Oracle-guided synthesis}
        \item Pretpostavlja da imamo implementaciju programa koji želimo da sintetišemo - uzor
        \item Uzor se tretira kao crna kutija
        \item Novi test primeri se kreiraju generišući proizvoljne ulaze a od uzora se dobijaju odgovarajući izlazi za svaki prosleđeni ulaz
    \end{itemize}
\end{frame}

\begin{frame}{CEGIS - Stohastička superoptimizacija}
    \begin{itemize}
        \item Pretražuje se prostor programa i traži se brži ili efikasniji ekvivalent polaznog programa
        \item Takođe se pretpostavlja da imamo implementaciju programa kao specifikaciju
        \item Koristi se \emph{MCMC} (eng. \emph{Markov-chain Monte Carlo sampling}) kao mera udaljenosti
    \end{itemize}
\end{frame}

\begin{frame}{CEGIS - Enumerativna pretraga}
    \begin{itemize}
        \item Za specifikaciju se koristi konačan skup test primera
        \item Pretpostavja se da je dostupna gramatika koja opisuje ciljani jezik (\texttt{add(x, sub(x,y))})
        \item Programi se dele prema dubini
        \item Sinteza kreće od dubine $0$ i numerišu se svi programi na toj dubini
        \item Na dubini $k$, ispituju se svi programe koji imaju oblik \texttt{operacija(a,b)}, gde su $a$ i $b$ bilo koji izrazi dubine $k-1$
    \end{itemize}
\end{frame}
