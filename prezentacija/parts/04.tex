
\begin{frame}{CEGIS - Sinteza vodjena uzorom}
    \begin{itemize}
        \item \emph{Oracle-guided synthesis}
        \item Pretpostavlja postojanje uzora (npr. imlementacija programa)
        \item Biblioteka komponenti za kreiranje programa
		\item Primer:      
				program(x,y):\\
				o1 = add(x, y)\\
				o2 = add(o1, y)\\
				o3 = sqrt(o1)\\
				return o3\\

        \item Faza verifikacije:\\Da li postoji program P', različit od kandidata za rešenje P, koji takođe zadovoljava sve test primere, ali se na nekom ulazu z razlikuje od P?
        \item Povratni korak - razmatra novodobijeni ulaz z
        \item Faza validacije - potvrda da program zadovoljava sve ulaze
    \end{itemize}
\end{frame}

\begin{frame}{CEGIS - Stohastička superoptimizacija}
    \begin{itemize}
        \item Traži se brži ili efikasniji ekvivalent polaznog programa
        \item Faza sinteze: novi program dobijamo primenom MCMC
        \item novi program prihvatamo sa određenom verovatnoćom
        \item verovatnoća je veća što su polazni i ciljni program sličniji
        \item Faza verifikacij: proverama da li su ciljni i program kandidat isti
        \item Povratni korak: poredimo prthodno prihvaćeni i novodobijeni program
        \item određujemo koji od njih dalje razamtramo
    \end{itemize}
\end{frame}

\begin{frame}{CEGIS - Enumerativna pretraga}
    \begin{itemize}
        \item Specifikacija - konačan skup test primera
        \item Gramatika opisuje ciljani jezik (\texttt{add(x, sub(x,y))})
        \item Faza sinteze: pretražuje sve moguće programe
        \item Faza verifikacije: proverava da li program zadovoljava sve test primere
        \item Povratni korak: razmatramo samo različite progame
        \item različiti programi daju različite rezultate na istom test primer
        \item Sinteza kreće od dubine $0$ i numerišu se svi programi na toj dubini
        \item Na dubini $k$, ispituju se svi programe koji imaju oblik \texttt{operacija(a,b)}, gde su $a$ i $b$ bilo koji izrazi dubine $k-1$
    \end{itemize}
\end{frame}
