\section{Zaključak}
\label{sec:zakljucak}

Razvojem tehnika automatske sinteze programa postavlja se pitanje da li će programeri moći da prestanu da govore računarima \textbf{kako} da rade, već da se fokusiraju na to da im kažu \textbf{šta} treba da urade. Ova oblast još uvek nije dovoljno razvijena da bi se koristila za razvijanje realnih, velikih aplikacija. Potrebno je još rada da bi se došlo do toga. Najveći potencijal ima induktivna sinteza programa.
Iako teorijski deluje da ovaj pristup nije dovoljno efikasan, te da će se izvršavati predugo, CEGIS je, kao vodeći predstavnik ove grupe, je u praksi pokazao neočekivano dobre rezultate. Uprkos tome, i uspešno sintetisanje manjih programa može značajno da olakša rad programerima. S obzirom na to da najveći deo vremena programeri provedu pišući manje delove koda, pa i automatska sinteza samo tih delova može značajno da ubrza njihov rad.
