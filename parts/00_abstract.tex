\abstract

Sinteza programa je oblast koja se bavi automatskim generisanjem programa. Korisnik na neki način opiše željeni program, a zadatak sintezera je da ga generiše tako da on zadovoljava zadata ograničenja. Sintezer može svom zadatku da pristupi na različite načine, a neki od tih načina su detaljnije opisani u radu. Primene sinteze programa su brojne, od priprema podataka, popravka i dobijanja sugestija prilikom kodiranja do predloga mogućih optimizacija. Kao jedan od značajnijih pristupa sintezi, detaljno je opisana tehnika CEGIS. To je pristup sintezi koji u svojoj osnovi sadrži iterativno generisanje kandidata za traženi program i proveru da li taj kandidat zadovoljava date uslove. Ukoliko ih zadovoljava, on se prosleđuje korisniku kao krajnje rešenje.
