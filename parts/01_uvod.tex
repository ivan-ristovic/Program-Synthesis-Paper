\section{Uvod}
\label{sec:uvod}

Danas sve veći broj ljudi ima pristup pametnim uređajima, poput tableta, telefona i računara. Međutim, samo mali procenat ljudi ima adekvatna znanja za programiranje takvih uređaja. Standardni način programiranja ovih uređaja se sastoji od dizajniranja algoritma koji rešava problem i njegove implementacije. Automatska sinteza programa ima potencijal da promeni generalni pristup programiranja pametnih uređaja. Ona će omogućiti nestručnim licima da programiraju uređaje bez opisa algoritama i implementacije istih. Ideja je da korisnici daju opis željenih funkcionalnosti programu za automatsko generisanje koda (u daljem tekstu \emph{sintezeru}), a on će im automatski generisati neophodnu implementaciju.

Poslednjih nekoliko decenija došlo je do značajnog pomaka u razvoju automatskog rezonovanja, naročito u unapređivanju SAT (eng. \emph{Propositional satisfiability problem}) i SMT (eng. \emph{Satisfiability modulo theories}) rešavača [?], koji su sad u mogućnosti da reše i neke industrijske probleme [?]. Ovaj napredak u automatskom rezonovanju dao je vetar u leđa razvoju programske sinteze koja svoja rešenja zasniva na logici prvog i drugog reda i automatskom dokazivanju teorema [?].

U ovom radu osvrnućemo se na neke primene programske sinteze, najveće izazove koji se u njoj javljaju, i neke tehnike zasnovane na CEGIS-u (eng. \emph{Counterexample-guided inductive synthesis}) koje su podstakle razvoj novih alata u ovoj oblasti.
