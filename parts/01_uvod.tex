\section{Uvod}
\label{sec:uvod}

U današnje vreme, pristup modernim tehnologijama je svima omogućen. Potražnja za softverom je sve veća, a samo mali procenat ljudi ima adekvatna
znanja za njegovo programiranje. Standardni način programiranja se sastoji od dizajniranja algoritma koji rešava problem i njegove implementacije. Automatska sinteza programa ima potencijal da promeni generalni pristup implementaciji programa. Ovakav način kreiranja softvera bi mogao da omogući i manje stručnim licima da programiraju bez dubokog poznavanja algoritama, struktura podataka i optimizacija.
Ideja je da korisnici daju opis željenih funkcionalnosti programu za automatsko generisanje koda (u daljem tekstu \emph{sintezeru}), a on će im automatski generisati neophodnu implementaciju.

Poslednjih nekoliko decenija došlo je do značajnog pomaka u razvoju automatskog rezonovanja, naročito u unapređivanju SAT (eng. \emph{Propositional Satisfiability Problem}) i SMT (eng. \emph{Satisfiability Modulo Theories}) rešavača \cite{SMT}, koji su sad u mogućnosti da reše i neke industrijske probleme \cite{PSE}. Ovaj napredak u automatskom rezonovanju dao je vetar u leđa razvoju programske sinteze koja svoja rešenja zasniva na logici prvog i drugog reda i automatskom dokazivanju teorema.

U ovom radu osvrnućemo se na neke primene programske sinteze, najveće izazove koji se u njoj javljaju, i neke tehnike zasnovane na CEGIS-u (eng. \emph{Counterexample-Guided Inductive Synthesis}). Navedene oblasti su podstakle razvoj novih alata i doprinele razvoju sinteze programa kao posebnog pravca savremenog računarstva.
