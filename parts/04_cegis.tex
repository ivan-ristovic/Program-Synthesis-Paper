\section{CEGIS}
\label{sec:cegis}

What is CEGIS?
Synthesis tasks often have the same structure: an implementation is sought that behaves correctly under all possible inputs (with the help of some extra variables, i.e. helper variables).

It is absolutely legal to pass such a term to an SMT solver like Z3. A big drawback is the universal quantifier though.

For many real world problems it is not necessary to consider all inputs to derive an implementation that behaves correctly for all of them. Following this observation, the problem was just moved to another position: which is the minimal subset of inputs one have to consider to ensure a correct synthesis?

This is the point where CEGIS comes into play. CEGIS is a loop looking for exactly this minimal subset of inputs and performing the implementation synthesis as a "by-product". Therefore CEGIS uses one satisfiability solver to generate new implementations based on all the inputs considered so far (starting with zero); and another one to generate counter examples that uncover incorrect behavior in the latest synthesised implementation. Eventually there will be no more implementations possible, i.e. the specification is not realisable, or no more counter examples possible, i.e. the latest implementation must be correct.

This CEGIS library works with the SMT solver Z3 and requires insight in the synthesis task to be executed, as it has to be specified which variables belong to implementation, inputs, etc. Boundary conditions are to be specified manually as well.
