\section{Primene}
\label{sec:Primene}

\subsection{Priprema podataka}
\label{subsec:PripremaPodataka}

Priprema podataka predstavlja proces čišćenja, transformacije i pripreme podataka iz polu-struktuiranog formata u format pogodan za analizu i prezentovanje. Procenjuje se da se 80\% vremena potroši na dovodjenje podataka u oblik pogodan za primenu algoritama iz mašinskog učenja ili istraživanja podataka radi izvlačenja korisnih zaključaka.

Proces pripreme podataka često obuhvata sledeće korake:
\begin{itemize}
  \item izvlačenje
  \item transformacija
  \item formatiranje
\end{itemize}

Programiranje vođeno primerima  (\url{https://www.microsoft.com/en-us/research/wp-content/uploads/2016/12/pbe16.pdf}) čini čitav ovaj proces bržim [44].


\subsection{Grafika}
\label{subsec:Grafika}

Programski opis grafičkih objekata omogućava dinamičku geometriju koja dovodi do bržih proračuna koordinata zavisnih tačaka od tačaka koje imaju slobodne koordinate. Ovo omogućava interaktivne izmene i efikasne animacije. Tehnike sinteze programa mogu uspešno generisati rešenja geometrijskih problema srednješkolske težine [48].

Slike i crteži (u daljem tekstu grafike) nekada sadrže ponovljene šablone, teksture ili objekte. Konstrukcija takvih grafika zahteva pisanje skriptova ili copy-paste operacija, što može biti jako neprijatno i podložno greškama. Korišćenjem sinteze programa, moguće je da korisnik prikaže par primera i ostavi posao sintezeru da predvidi naredne objekte u nizu [21]. Štaviše, korišćenjem grafičkog interfejsa za domen vektorske grafike, moguće je interaktivno omogućiti korisniku crtanje isključivo pomoću grafičkih alata a generisanje programa ostaviti sintezeru.


\subsection{Popravka koda}
\label{subsec:PopravkaKoda}

Postoji mnogo tehnika sinteze napravljenih specifično za problem popravke koda. [26, 60, 99, 130]. Za dat program P i specifikaciju $\phi$, problem popravke zahteva računanje modifikacija programa P koje stvaraju nov program P’ takav da zadovoljava $\phi$. Osnovna ideja ovih tehnika je da se prvo ubace alternativni izbori za izraze u programu, a onda tehnikama sinteze nađu zamene ili modifikacije izraza pronađu izrazi koji program dovode u oblik koji zadovoljava $\phi$.


\subsection{Sugestije prilikom kodiranja}
\label{subsec:SugestijePrilikomKodiranja}


\subsection{Modelovanje}
\label{subsec:Modelovanje}

\subsection{Superoptimizacija}
\label{subsec:Superoptimizacija}

\subsection{Konkurentno programiranje}
\label{subsec:KonkurentnoProgramiranje}
