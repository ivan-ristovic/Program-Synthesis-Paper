\section{Izazovi}
\label{sec:Izazovi}

Pisanje programa koji može da sintetiše drugi program je oblast koja je tek na početku svog razvoja. Danas, napisati čak i jednostavan sintezer nije lako. Naime, posmatrano sa visokog nivoa, problem sinteze se može razložiti na dva potproblema čije je efikasno rešavanje njen najveći izazov. Ti potproblemi su:

\begin{itemize}
    \item Definisanje specifikacija željenog programa,
    \item Pretraživanje prostora mogućih programa u potrazi za onim koji zadovoljava definisane specifikacije.
\end{itemize}

Prostor programa se povećava eksponencijalno brzo u odnosu na ve\-li\-či\-nu željenog programa. Zbog toga postoje različiti pristupi njegovog pretraživanja, a neke od tih tehnika su opisane u poglavlju \ref{subsec:ProstorPrograma}.

\subsection{Definisanje specifikacija}
\label{subsec:DefinisanjeSpecifikacija}

Generisani program treba da se ponaša na način na koji to korisnik definiše. Međutim, precizno definisanje zahteva je zapravo mnogo teže nego što izleda na prvi pogled. Postoje različiti načini za definisanje zahteva, npr. korišćenjem formalnih logičkih izraza, zadavanjem primera ulaza i izlaza programa ili definisanjem neformalnim metodama.

Formalno definisanje zahteva je ponekad i komplikovanije od pisanja samog programa. Nasuprot tome, neformalne metode su mnogo prirodnije korisniku, ali dovode do drugih problema. Na primer, neka se željeni program definiše na osnovu primera njegovog ulaza i izlaza na sledeći način: \\
\(\emph{“John Smith”} \rightarrow \emph{“Smith, J.”}\).\\
Ovaj program vrši nama intuitivnu transformaciju niski, ali, na primer, da bi se on automatski generisao korišćenjem \emph{FlashFill} \cite{FlashFill} programa, potrebno je pretražiti prostor programa koji sadrži milione mogućih rešenja. Problem je u tome što programi nemaju ljudsku intuiciju, već se pre\-pri\-la\-go\-đa\-va\-ju datim primerima ulaza i izlaza.

Većina programa koji se danas koriste su previše komplikovani da bi se u potpunosti opisali bilo formalnim bilo neformalnim metodama. Čak i kad bi se to uspelo, opis programa bi mogao da bude toliko obiman koliko i sama njegova implementacija. Kako bi sinteza ovakvih, realnih programa bila moguća, potrebno je omogućiti korisniku da on na početku definiše željeni program do neke tačke, a da kasnije tokom sinteze, interaktivno sa računarom, postepeno dolazi do rešenja.

Upravo ovakvu naprednu pretragu koristi gorepomenuti program \emph{FlashFill}. Tokom pretrage, on uključuje dodatnu komunikaciju sa korisnikom. Ovako on usmerava pretragu, te na kraju ipak uspeva da nađe rešenje u realnom vremenu.


\subsection{Pretraživanje prostora programa}
\label{subsec:ProstorPrograma}

Svaka uspešna sinteza programa vrši neki vid pretrage prostora mo\-gu\-ćih programa (eng. \emph{search space}). Prostor mogućih programa predstavlja skup koji sadrži sve moguće programe koji se mogu napisati, a pretraga ovog skupa znači nalaženje programa koji zadovoljava specifikacije. Ovo je težak kombinatorni problem. Broj mogućih rešenja raste eksponencijalno sa veličinom programa, te pretraga svih kandidata nije moguća u realnom vremenu. Potrebno je pažljivo vršiti odsecanja dela prostora pretrage kako bi se došlo do rešenja u realnom vremenu.

Tehnike pretrage se mogu zasnivati na enumerativnoj pretrazi, dedukciji, tehnikama sa ograničenjima, induktivnim i statističkim metodama. U praksi se uglavnom koristi neka od njihovih kombinacija, te ih je teško razgraničiti kao posebne metode pretrage.


\subsubsection{Enumerativna pretraga}
\label{subsubsec:Enumerative}

Tehnike enumerativne pretrage za sintezu programa su se pokazale kao jedne od najefikasnijih tehnika za generisanje malih programa. Razlog ove efikasnosti je u pametnim tehnikama \emph{čišćenja} (eng. \emph{pruning}) u prostoru programa koji se pretražuje. Glavna ideja je da se prvo na neki način opiše prostor pretrage u kome se nalazi željeni program. To može da se postigne korišćenjem metapodataka kao što su veličina programa ili njegova složenost. Kada se mogući programi numerišu po osobinama, mogu da se odmah odbace oni koji ne zadovoljavaju prethodno definisane specifikacije.

Kako se na osnovu pretpostavki vrše velika odsecanja, moguće je doći do gubitka nekog od mogućih rešenja usled pogrešnog numerisanja na početku. Zbog ovoga je enumerativna tehnika poluodlučiva, tj. ne garantuje pronalazak rešenja ukoliko ono postoji. Međutim, u opštem slučaju je upotrebljiva i daje dobre rezultate i to relativno brzo.


\subsubsection{Deduktivna pretraga}
\label{subsubsec:Deductive}

Deduktivna sinteza programa je tradicionalni pogled na sintezu programa. Ovakvi pristupi pretpostavljaju da postoji celokupna formalna spe\-ci\-fi\-ka\-ci\-ja željenog programa. Ovo je vrlo jaka pretpostavka imajući u vidu da ta specifikacija može da bude veoma velika ukoliko je program kompleksan. Rešenje se sintetiše postupkom dokazivanja teorema, logičkim zaključivanjem i razrešavanjem ograničenja.

Deduktivna pretraga je pretraga odozgo nadole. Koristi tehniku podeli-pa-vladaj (eng. \emph{divide-and-conquer}). Program se sintetiše tako što se prvo podeli na potprobleme tako da svaki od njih ima svoju specifikaciju. Rekurzivno se obrade potproblemi, a zatim iskombinuju podrešenja kako bi se dobilo glavno rešenje.

Deljenje problema na potprobleme koji mogu da se sintetišu odvojeno nije moguće u opštem slučaju. Ovo zavisi od prirode problema. U ovom slučaju se deduktivna pretraga može iskombinovati sa enumerativnom. Kada deduktivna pretraga više ne može da razloži problem, enumerativnom pretragom (koja je odozdo - na gore) tada treba pretražiti prostor rešenja potproblema, a nakon toga spojiti dobijene rezultate.

Deduktivna pretraga može lako da zaključi vrednosti konstanti u programu. To je bitno jer ukoliko program sadrži veliki broj konstanti, sama enumerativna pretraga bi provela mnogo vremena pokušavajući da pogodi njihove prave vrednosti.


\subsubsection{Tehnike sa ograničenjima}
\label{subsubsec:ConstraintSolving}

Mnoge uspešne tehnike sinteze programa u svojoj osnovi sadrže tehnike prilagođavanja datim ograničenjima (eng. \emph{constraint solving}). One se sastoje od dva velika koraka:
\begin{itemize}
    \item \emph{Generisanje ograničenja} - U opštem slučaju, kada se kaže pri\-la\-go\-đa\-va\-nje ograničenjima, misli se na pronalaženje modela za formulu koja opisuje željeni program. Osnovna ideja je da se specifikacija programa kao i njegova dodatna ograničenja zapišu u jednoj logičkoj formuli. Uglavnom se tom prilikom u formulu dodaju i pretpostavke o rešenju.
    \item \emph{Razrešavanje ograničenja} - Formula u kojoj su zapisana ograničenja često sadrži kvantifikatore i nepoznate drugog reda. Ona se prvo transformiše u oblik pogodan za neki od rešavača, na primer za SAT ili SMT rešavač. Na ovaj način se problem pretrage svodi na problem ispitivanja zadovoljivosti prosleđene formule. Svaki nađeni model za tu formulu predstavlja jedno moguće rešenje koje zadovoljava data ograničenja.
\end{itemize}


\subsubsection{Induktivna pretraga}
\label{subsubsec:Induktivna}

Induktivna pretraga je veoma opšti pojam. Ona se, na primer, može smatrati kao nadogradnja tehnike pretrage sa ograničenjima (videti \ref{subsubsec:ConstraintSolving}) \cite{SGS}. Prilikom svake iteracije se generišu ograničenja, rešavačem se dođe do mogućeg rešenja a zatim se ispita da li je ono zadovoljavajuće kao opšte rešenje.

Sa druge strane, induktivna pretraga može da koristi tehnike mašinskog učenja. To može da radi na razne načine, a jedan od takvih pristupa je sinteza sa aktivnim učenjem \cite{SGS}. \emph{Aktivno učenje} (eng. \emph{Active learning}) je poseban vid mašinskog učenja gde je algoritmu za učenje dopušteno da sam vrši selekciju podataka na osnovu kojih će da uči. Ovakav sintezer koristi induktivni algoritam za učenje, koji u sebi često poziva i deduktivne procedure. Iako kombinuje oba pristupa, sinteza sa aktivnim učenjem se u literaturi predstavlja kao primer induktivne sinteze.

Dobra strana induktivnog pristupa jeste što je dovoljno fleksibilan da može da radi i sa nepotpunim specifikacijama problema. Međutim, može se desiti da rezultat ne zadovolji očekivanja korisnika ukoliko neki bitni slučajevi nisu pokriveni specifikacijom datom primerima ulaza i izlaza.

Jedan primer induktivne pretrage je CEGIS. Ova tehnika će biti detaljno opisana u poglavlju \ref{sec:cegis}.


\subsubsection{Statistička pretraga}
\label{subsubsec:Statistical}

Postoji veliki broj metoda koje se koriste za pretragu prostora programa, a koriste neku vrstu statistike kako bi došle do rešenja. Neke od tehnika koje se mogu koristiti su tehnike mašinskog učenja, genetsko programiranje i probabilističko zaključivanje.

\emph{Mašinsko učenje} - Tehnike mašinskog učenja mogu doprineti ostalim pretragama uvodeći verovatnoću u čvorove grananja prilikom pretrage. Vrednosti verovatnoće se uglavnom generišu pre sinteze programa: tokom treninga ili, na primer, na osnovu datih primera ulaza i izlaza \cite{MLinProgramSyntesys}.

\emph{Genetsko programiranje} - Genetsko programiranje je metod inspirisan biološkom evolucijom. Sastoji se od održavanja populacije programa, njihovog ukrštanja i mogućih mutacija. Ispituje se u kojoj meri svaka jedinka zadovoljva. One jedinke koje bolje odgovaraju rešenju nastavljaju da evoluiraju. Uspeh genetskih algoritama zavisi od funkcije zadovoljivosti, čiji dobar izbor predstavlja najteži problem ove tehnike.

\emph{Probabilističko zaključivanje} (eng. \emph{Probabilistic inference}) - Ova te\-hni\-ka dolazi do rešenja nadograđivanjem početnog programa. Dodaju se sitne izmene, jedna po jedna, i proverava se da li takav promenjen program zadovoljava specifikacije.
