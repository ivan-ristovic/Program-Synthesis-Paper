\section{Izazovi}
\label{sec:Izazovi}

Pisanje programa koji može da sintetiše drugi program je veliki izazov. Naime, ovaj problem sinteze se može razložiti na dva potproblema:

\begin{itemize}
  \item Definisanje specifikacija željenog programa,
  \item Pretraživanje prostora mogućih programa u potrazi za onim koji zadovoljava definisane specifikacije
\end{itemize}

Prostor programa se povećava eksponencijalno brzo, te postoje različite tehnike za njegovo pretraživanje.


\subsection{Definisanje specifikacija}
\label{subsec:DefinisanjeSpecifikacija}

Generisani program treba da se ponaša na način koji to korisnik definiše. Međutim, precizno definisanje zahteva predstavlja velik izazov. Postoje različiti načini na koje se to može postići; od formalnih logičkih izraza do neformalnih opisa ili primera ulaza/izlaza programa.

Formalno definisanje zahteva često izgleda komplikovano (možda čak deluje i komplikovanije neko pisanje samog programa). Nasuprot tome, neformalne metode su mnogo prirodnije korisniku, ali dovode do drugih problema. Na primer, neka se željeni program definiše na osnovu primera njegovog ulaza i izlaza \emph{“John Smith” -> “Smith, J.”}. Ovaj program vrši nama intuitivnu transformaciju niski, ali da bi se on automatski generisao korišćenjem FlashFill \cite{FlashFill} programa, on treba da pretraži prostor koji sadrži milione mogućih rešenja. Problem je u tome što programi nemaju ljudsku intuiciju, već se preprilagođavaju datim primerima ulaza i izlaza. Uz dodatnu komunikaciju sa korisnikom FlashFill usmerava pretragu te na kraju uspeva da nađe rešenje.

Većina programa koji se danas koriste su previše komplikovani da bi se potpuno opisali bilo formalnim bilo neformalnim metodama. Čak i ako bi se to nekako uspelo, opis programa bi bio toliko obiman kao i sama implementacija programa. Kako bi senteza ovakvih, realnih programa bila moguća, potrebno je omogućiti korisniku da ne definiše program do kraja, već da se interaktivno sa njim tokom sinteze postepeno dolazi do rešenja.


\subsection{Prostor programa}
\label{subsec:SearchTechnique}

Svaka uspešna sinteza programa vrši neki vid pretrage prostora mogućih programa. Ovo je težak kombinatorni problem. Broj mogućih rešenja raste eksponencijalno sa veličinom programa, te pretraga svih kandidata nije moguća u realnom vremenu. Potrebno je pažljivo vršiti odsecanja dela prostora pretrage kako bi se došlo do rešenja u realnom vremenu.

Tehnike pretrage se mogu zasnivati na enumerativnoj pretrazi, dedukciji, constraint solving, statističkim tehnikama, kao i na kombinaciji nekih od nih.


\subsubsection{Enumerativna pretraga}
\label{subsubsec:Enumerative}

An enumerative search technique enumerates programs
in the underlying search space in some order and for each program
checks whether or not it satisfies the synthesis constraints. While this
might appear simple, it is often a very effective strategy. A naïve implementation
of enumerative search often does not scale. Many practical
systems that leverage enumerative search innovate by developing various
optimizations for pruning the search space or by ordering it.


\subsubsection{Deduktivna pretraga}
\label{subsubsec:Deductive}

Specifications on the formal end of this spectrum
(traditionally required by deductive synthesis techniques) often appear
to the user as complex as writing the program itself. 

The deductive top-down search [113] follows the standard
divide-and-conquer technique, where the key idea is to recursively reduce
the problem of synthesizing a program expression e of a certain kind and
that satisfies a certain specification fiii to simpler sub-problems (where
the search is either over sub-expressions of e or over sub-specifications
of fiii), followed by appropriately combining those results. The reduction
logic for reducing a synthesis problem to simpler synthesis problems
depends on the nature of the involved expression e and the inductive
specification fiii. In particular, if e is of the form F(e1, e2), the reduction
logic leverages the inverse semantics of F to push constraints on e down
through the grammar into constraints on e1 and e2.
While enumerative search is bottom-up (i.e., it enumerates smaller
sub-expressions before enumerating larger expressions), the deductive
search is top-down (i.e., it fixes the top-part of an expression and
then searches for its sub-expressions). Enumerative search can be seen
as finding a programmatic path (within an underlying grammar that
connects inputs and outputs) starting from the inputs to outputs.
Deduction does the same, but it searches for the programmatic path in
a backward direction starting from the outputs leveraging the operator
inverses. If the underlying grammar allows for a rich set of constants,
the bottom-up enumerative search can get lost in simply guessing the
right constants. On the other hand, the top-down deductive technique
can deduce constants based on the accumulated constraints as the last
step in the search process.


\subsubsection{Constraint Solving}
\label{subsubsec:ConstraintSolving}

The constraint solving based techniques [132, 135]
involve two main steps: constraint generation, and constraint resolution.

Constraint generation refers to the process of generating a logical
constraint whose solution will yield the intended program. Generating
such a logical constraint typically requires making some assumption
about the control flow of the unknown program and encoding that
control flow in some manner. Three different kinds of methods have been
used in the past for constraint generation: invariant-based, path-based,
and input-based. On one extreme, we have invariant-based methods
that generate constraints that faithfully assert that the program satisfies
the given specification [133].
Such methods also end up synthesizing an inductive proof of correctness
in addition to the program itself. A disadvantage of such methods
is that the generated constraints may be very sophisticated since the
inductive invariants are often much more complicated and over a richer
logic than the program itself. On the other extreme, we have inputbased
methods that generate constraints that assert that the program
satisfies the given specification on a certain collection of inputs [132].
Such constraints are usually much simpler in nature than the ones
generated by the invariant-bases method. Unless paired with a sound
counterexample guided inductive synthesis strategy (CEGIS), described
in §3.2, this method trades off soundness for efficiency. A middle ground
is achieved by path-based methods that generate constraints that assert
that the program satisfies the given specification on all inputs that
execute a certain set of paths [134]. Compared to input-based methods,
these methods may achieve a faster convergence, if paired up with an
outer CEGIS loop.
Constraint solving involves solving the constraints outputted by the
constraint generation phase. These constraints often involve secondorder
unknowns and universal quantifiers. A general strategy is to first
reduce the second-order unknowns to first-order unknowns and then
eliminate universal quantifiers, and then solve the resulting first-order
quantifier-free constraints using an off-the-shelf SAT/SMT solver. The
second-order unknowns are reduced to first-order unknowns by use of
templates. The universal quantifiers can be eliminated using a variety
of strategies including Farkas lemma, cover algorithms, and sampling.


\subsubsection{Statistička pretraga}
\label{subsubsec:Statistical}

Postoji veliki broj statističkih metoda koje mogu da se upotrebe za pretragu. Neke od njih su:
\begin{itemize}
  \item Mašinskog učenje
  Machine learning techniques can be used to augment other search
methodologies based on enumerative search or deduction by providing
likelihood of various choices at any choice point. One such choice point
is selection of a production for a non-terminal in a grammar that
specifies the underlying program space. The likelihood probabilities can
be function of certain cues found in the input-ouput examples provided
by the user or the additionally available inputs [89]. These functions
are learned in an offline phase from training data.

  
  \item Genetičkog programiranje
  Genetic programming is a program synthesis method inspired by
biological evolution [72]. It involves maintaining a population of individual
programs, and using that to produce program variants by
leveraging computational analogs of biological mutation and crossover.
Mutation introduces random changes, while crossover facilitates sharing
of useful pieces of code between programs being evolved. Each variant’s
suitability is evaluated using a user-defined fitness function, and
successful variants are selected for continued evolution. The success
of a genetic programming based system crucially depends on the fitness
function. Genetic programming has been used to discover mutual
exclusion algorithms [68] and to fix bugs in imperative programs [146]


  \item MCMC sampling
  MCMC sampling has been used to search for a desired program
starting from a given candidate. The success crucially depends on
defining a smooth cost metric for Boolean constraints. STOKE [124], a
superoptimization tool, uses Hamming distance to measure closeness
of generated bit-values to the target on a representative test input set,
and rewards generation of (almost) correct values in incorrect locations.


  \item Genetičkog programiranje
Probabilistic inference has been used to evolve a given program by
making local changes, one at a time. This relies on modeling a program
as a graph of instructions and states, connected by constraint nodes.
Each constraint node establishes the semantics of some instruction by
relating the instruction with the state immediately before the instruction
and the state immediately after the instruction [45]. Belief propagation
has been used to synthesize imperative program fragments that execute
polynomial computations and list manipulations [62].

\end{itemize}
