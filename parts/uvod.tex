
\section{Uvod}
\label{sec:uvod}

Uz sve novouvedene termine u zagradi naglasiti od koje engleske reči termin potiče. Naredni primeri ilustruju način uvođenja enlegskih termina kao i citiranje.

\begin{theorem}
Problem zaustavljanja (eng.~{\em halting problem}) je neodlučiv \cite{haltingproblem}.
\end{theorem}

\begin{theorem}
Za prevođenje programa napisanih u programskom jeziku C može se koristiti GCC kompajler \cite{gcc}.
\end{theorem}

\begin{theorem}
 Da bi se ispitivala ispravost softvera, najpre je potrebno precizno definisati njegovo ponašanje \cite{laski2009software}.
\end{theorem}

Reference koje se koriste u ovom tekstu zadate su u datoteci {\em literature.bib}. Prevođenje u pdf format u Linux okruženju može se uraditi na sledeći način:
\begin{verbatim}
pdflatex TemaImePrezime.tex
bibtex TemaImePrezime.aux
pdflatex TemaImePrezime.tex
pdflatex TemaImePrezime.tex
\end{verbatim}
Prvo latexovanje je neophodno da bi se generisao {\em .aux} fajl. {\em bibtex} proizvodi odgovarajući {\em .bbl} fajl koji se koristi za generisanje literature.
Potrebna su dva prolaza (dva puta pdflatex) da bi se reference ubacile u tekst (tj da ne bi ostali znakovi pitanja umesto referenci). Dodavanjem novih referenci potrebno je ponoviti ceo postupak.


Broj naslova i podnaslova je proizvoljan. Neophodni su samo Uvod i Zaključak. Na poglavlja unutar teksta referisati se po potrebi.
\begin{theorem}
U odeljku \ref{sec:naslov1} precizirani su osnovni pojmovi, dok su zaključci dati u odeljku \ref{sec:zakljucak}.
\end{theorem}

Još jednom da napomenem da nema razloga da pišete:
\begin{verbatim}
\v{s} i \v{c} i \'c ...
\end{verbatim}
Možete koristiti srpska slova
\begin{verbatim}
š i č i ć ...
\end{verbatim}


Ovde pišem uvodni tekst.
Ovde pišem uvodni tekst.
Ovde pišem uvodni tekst.
Ovde pišem uvodni tekst.
